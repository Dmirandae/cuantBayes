
\begin{frame}{Likelihood as a Relative Measure}
  \begin{itemize}
    \item Likelihood emphasizes the relativity of evidence.
    \item The meaningfulness of the likelihood of a parameter value is realized when compared to alternative values.
  \end{itemize}
\end{frame}

\begin{frame}{Arbitrary Scaling in Likelihood Profile}
  \begin{itemize}
    \item Likelihood profiles use arbitrary scaling on the y-axis.
    \item The arbitrary constant 'c' allows the profile values to take any magnitude, limiting insights into a single parameter value.
  \end{itemize}
\end{frame}

\begin{frame}{Importance of Comparison}
  \begin{itemize}
    \item Understanding parameters relies on comparing the likelihood of one value with another.
    \item The constant of proportionality becomes inconsequential in this comparative analysis.
  \end{itemize}
\end{frame}

\begin{frame}{Likelihood Ratios}
  \begin{itemize}
    \item Likelihood ratios are crucial for comparing evidence for alternative parameter values.
    \item The ratio \( \frac{L(\theta_1|y)}{L(\theta_2|y)} \) encapsulates the comparative evidence.
  \end{itemize}
\end{frame}

\begin{frame}{Interpreting Likelihood Ratios}
  \begin{itemize}
    \item The likelihood principle states that all information about alternative parameter values is in the likelihood ratio.
    \item Interpretation involves understanding how the data supports one parameter value relative to another.
  \end{itemize}
\end{frame}

\begin{frame}{Logarithmic Transformation}
  \begin{itemize}
    \item The natural logarithm of the likelihood ratio is often used to quantify support.
    \item Formally defined as the support for one parameter value over another, conditional on the data.
  \end{itemize}
\end{frame}

\begin{frame}{Basis for Comparison}
  \begin{itemize}
    \item The ratio of likelihoods or the difference between log likelihoods establishes the foundation for evaluating evidence.
    \item Particularly important in ecological problems for determining the parameter value that garners maximum support.
  \end{itemize}
\end{frame}

\begin{frame}{Maximum Likelihood Estimate (MLE)}
  \begin{itemize}
    \item MLE involves finding the parameter value that maximizes the likelihood function.
    \item Analytical methods for simple models, numerical techniques for complex scenarios.
  \end{itemize}
\end{frame}


