
\documentclass{beamer}

\begin{document}

\title{Likelihood}
\subtitle{Hobbs 2015}
\author{DRME}
\date{\today}

\begin{frame}
  \titlepage
\end{frame}

\begin{frame}{Probability in Biology: Hobbs 4.1}
  \begin{itemize}
    \item Example: Tadpole observation in a pond.
  \end{itemize}
\end{frame}

\begin{frame}{Example: Tadpole Observation}
  \textbf{Scenario:}
  \begin{itemize}
    \item Collecting data on the number of tadpoles per volume of water in a pond.
    \item Observed 14 tadpoles in a 1 L sample.
    \item {\bf{TRUE}} average number of tadpoles per liter of water in the pond is 23.
  \end{itemize}

  \textbf{First Observation:}
  \begin{itemize}
  \item It is Poisson
    \item Probability of observing 14 tadpoles: $P(y_1 = 14|\lambda = 23) = \text{Poisson}(y_1 = 14|\lambda = 23) = 0.0136$.
  \end{itemize}

  \textbf{Second Observation:}
  \begin{itemize}
    \item Probability of observing 34 tadpoles: $P(y_2 = 34|\lambda = 23) = \text{Poisson}(y_2 = 34|\lambda = 23) = 0.0069$.
  \end{itemize}

  \textbf{Joint Probability:}
  \begin{itemize}
    \item Assuming independence: Joint probability = $0.0136 \times 0.0069 = 9.38 \times 10^{-5}$.
  \end{itemize}
\end{frame}

\begin{frame}{Independence of Observations}
  \begin{itemize}
    \item Independence assumption: Knowledge of one observation tells us nothing about the other.
    \item Joint probability calculation extended to any number of independent observations.
  \end{itemize}
\end{frame}

\begin{frame}{Remarks}
  \begin{itemize}
    \item Probability calculations provide insights into the likelihood of observations given a fixed average.
    \item Independence assumption crucial for joint probability calculations.
    \item The Poisson distribution to model catching probabilities.
  \end{itemize}
\end{frame}

\begin{frame}{Probability in Biology: Hobbs 4.2}
  \begin{itemize}
    \item Investigating decomposition of leaf litter over time.
    \item Using a simple model of exponential decay: \(\mu_t = e^{-kt}\).
    \item Data: \(y_t\) - observed proportions, {\bf{modeled with a beta distribution}}.
    \item Parameters: \(k\) (decay rate) and \(\sigma^2\) (variance).
  \end{itemize}
\end{frame}

\begin{frame}{Beta Distribution for \(y_t\)}
  \begin{itemize}
    \item Model the probability density of \(y_t\) with a beta distribution:
    \[ y_t | \mu_t, \sigma^2 \sim \text{beta}(\alpha_t, \beta_t) \]
    \item Moment matching for \(\alpha_t\) and \(\beta_t\):
    \begin{align*}
      \alpha_t &= \mu_t^2 - \mu_t^3 - \mu_t \sigma^2 \\
      \beta_t &= \mu_t - 2\mu_t^2 + \mu_t^3 - \sigma^2 + \mu_t \sigma^2
    \end{align*}
  \end{itemize}
\end{frame}

\begin{frame}{Conditional on Decay Rate \(k\) and \(\sigma^2\)}
  \begin{itemize}
    \item Conditional on known, fixed decay rate \(k = 0.01 \, \text{day}^{-1}\) and known, fixed \(\sigma^2 = 6 \times 10^{-4}\):
    \item Calculate parameters for the beta distribution on day 30:
    \begin{align*}
      \alpha_{30} &= 236.33 \\
      \beta_{30} &= 82.68
    \end{align*}
  \end{itemize}
\end{frame}

\begin{frame}{Probability Density Calculation}
  \begin{itemize}
    \item Given \(y_{30} = 0.7\), calculate the probability density:
    \[ f(y_{30} = 0.7) = 4.040 \]
    \item Interpretation: The probability that 70\% of the mass remains at time \(t = 30\) is 4.040.
  \end{itemize}
\end{frame}

\begin{frame}{Remarks}
  \begin{itemize}
    \item The beta distribution to model decay over time.
    \item Moment matching provides a method for estimating distribution parameters.
  \end{itemize}
\end{frame}


\begin{frame}{Introduction to Likelihood}
  \begin{itemize}
    \item Likelihood  measures the support provided by the observed data for different values of the parameter in a statistical model.
    \item The likelihood function is the foundation of maximum likelihood estimation (MLE).
  \end{itemize}
\end{frame}

\begin{frame}{Likelihood Function}
  \begin{itemize}
    \item The likelihood function, denoted as \(L(\theta; \mathbf{x})\), represents the probability of observing the given data \(\mathbf{x}\) for various parameter values \(\theta\) in the model.
    \item The likelihood function is not a probability distribution but provides a basis for estimating parameters.
  \end{itemize}
\end{frame}

\begin{frame}{Likelihood Example: Coin Toss}
  \begin{itemize}
    \item Consider a simple example: coin toss.
    \item Let \(\theta\) be the probability of getting heads (\(\theta \in [0, 1]\)).
    \item If we observe \(k\) heads in \(n\) tosses, the likelihood function is given by the binomial distribution:
    \[ L(\theta; k, n) = \binom{n}{k} \theta^k (1 - \theta)^{n-k} \]
  \end{itemize}
\end{frame}

\begin{frame}{Interpretation of Likelihood}
  \begin{itemize}
    \item Likelihood is not a probability, but it measures the compatibility of the observed data with different parameter values.
    \item Larger likelihood values indicate a better fit of the model to the observed data.
    \item The goal is to find the parameter values that maximize the likelihood, known as maximum likelihood estimation (MLE).
  \end{itemize}
\end{frame}

\begin{frame}{Likelihood in MLE}
  \begin{itemize}
    \item Maximum Likelihood Estimation (MLE) aims to find the parameter values that maximize the likelihood function.
    \item MLE is a common method for estimating parameters in statistical models.
    \item It provides point estimates that make the observed data most probable under the assumed model.
  \end{itemize}
\end{frame}

\begin{frame}{Likelihood Example: Population Growth}
  \begin{itemize}
    \item Consider a simple population growth model: \(N_t = N_0 \cdot e^{rt}\), where \(N_t\) is the population size at time \(t\), \(N_0\) is the initial population size, \(r\) is the growth rate, and \(e\) is the base of the natural logarithm.
    \item Likelihood function: \(L(r | \mathbf{y})\), where \(\mathbf{y}\) is the observed population size over time.
  \end{itemize}
\end{frame}

\begin{frame}{Likelihood Example: Phylogenetic Trees}
  \begin{itemize}
    \item In evolutionary biology, likelihood is extensively used in phylogenetic analysis.
    \item Given a phylogenetic tree and DNA sequence data, the likelihood of observing the given sequences under different substitution models is calculated.
    \item MLE finds the tree and model parameters that maximize the probability of the observed data.
  \end{itemize}
\end{frame}


\section{Probability Distribution vs. Likelihood Function}
 
 

\begin{frame}{Probability Distribution vs. Likelihood Function}
  \begin{itemize}
    \item Key distinction: Treatment of parameters and data.
    \item In a probability distribution:
      \begin{itemize}
        \item Parameter is fixed.
        \item Data are random variables.
      \end{itemize}
    \item In a likelihood function:
      \begin{itemize}
        \item Data are fixed.
        \item Parameters are variable.
      \end{itemize}
  \end{itemize}
\end{frame}

\begin{frame}{Examining the Relationship Through Plots}
  \begin{itemize}
    \item Probability density function with a fixed $\theta$ and varying $y$:
      \begin{itemize}
        \item Area under the curve equals 1 (Figure 4.2.1A).
      \end{itemize}
    \item Probability density function with fixed $y$ and varying $\theta$:
      \begin{itemize}
        \item Likelihood profile obtained (Figure 4.2.1B).
        \item Area under the curve does not equal 1.
      \end{itemize}
  \end{itemize}
\end{frame}

\begin{frame}{Relationship for Discrete Data}
  \begin{itemize}
    \item Probability mass function $y|\theta$ with fixed $\theta$ and varying $y$:
      \begin{itemize}
        \item Sum of probabilities equals 1 (Figure 4.2.1C).
      \end{itemize}
    \item Probability mass function with fixed $y$ and varying $\theta$:
      \begin{itemize}
        \item Likelihood profile generated (Figure 4.2.1D).
        \item Sum of probabilities does not equal 1.
      \end{itemize}
  \end{itemize}
\end{frame}

\begin{frame}{Units of the Likelihood Profile}
  \begin{itemize}
    \item Units on the y-axis are arbitrary.
    \item Can be scaled to any quantity.
    \item Likelihood profile often scaled so that the peak equals 1.
    \item Scaling achieved by dividing all likelihoods by the maximum likelihood.
    \item Does not alter the relationship between likelihood and probability.
  \end{itemize}
\end{frame}

\begin{frame}{Illustration Through Figure 4.2.1}
  \begin{itemize}
    \item Figure 4.2.1 visually illustrates the subtle distinction.
    \item Parameter $\theta$ is not fixed in the likelihood framework.
    \item Likelihood functions do not define the probability or probability density of $\theta$.
  \end{itemize}
\end{frame}

\begin{frame}{Notation Considerations}
  \begin{itemize}
    \item Some authors use $L(\theta; y)$ instead of $L(\theta|y)$.
    \item Emphasizes that likelihood functions focus on the dependence of parameters on fixed data.
  \end{itemize}
\end{frame}

\begin{frame}{Remarks}
  \begin{itemize}
    \item Probability distributions and likelihood functions highlighted through plots.
    \item Probability and likelihood coincide {\bf{ONLY}} when parameters are treated as fixed.
  \end{itemize}
\end{frame}



 
 
 
\section{Maximum Likelihood Estimation}

 
\begin{frame}{Likelihood as a Relative Measure}
  \begin{itemize}
    \item Likelihood emphasizes the relativity of evidence.
    \item The meaningfulness of the likelihood of a parameter value is realized when compared to alternative values.
  \end{itemize}
\end{frame}

\begin{frame}{Arbitrary Scaling in Likelihood Profile}
  \begin{itemize}
    \item Likelihood profiles use arbitrary scaling on the y-axis.
    \item The arbitrary constant 'c' allows the profile values to take any magnitude, limiting insights into a single parameter value.
  \end{itemize}
\end{frame}

\begin{frame}{Importance of Comparison}
  \begin{itemize}
    \item Understanding parameters relies on comparing the likelihood of one value with another.
    \item The constant of proportionality becomes inconsequential in this comparative analysis.
  \end{itemize}
\end{frame}

\begin{frame}{Likelihood Ratios}
  \begin{itemize}
    \item Likelihood ratios are crucial for comparing evidence for alternative parameter values.
    \item The ratio \( \frac{L(\theta_1|y)}{L(\theta_2|y)} \) encapsulates the comparative evidence.
  \end{itemize}
\end{frame}

\begin{frame}{Interpreting Likelihood Ratios}
  \begin{itemize}
    \item The likelihood principle states that all information about alternative parameter values is in the likelihood ratio.
    \item Interpretation involves understanding how the data supports one parameter value relative to another.
  \end{itemize}
\end{frame}

\begin{frame}{Logarithmic Transformation}
  \begin{itemize}
    \item The natural logarithm of the likelihood ratio is often used to quantify support.
    \item Formally defined as the support for one parameter value over another, conditional on the data.
  \end{itemize}
\end{frame}

\begin{frame}{Basis for Comparison}
  \begin{itemize}
    \item The ratio of likelihoods or the difference between log likelihoods establishes the foundation for evaluating evidence.
    \item Particularly important in ecological problems for determining the parameter value that garners maximum support.
  \end{itemize}
\end{frame}

\begin{frame}{Maximum Likelihood Estimate (MLE)}
  \begin{itemize}
    \item MLE involves finding the parameter value that maximizes the likelihood function.
    \item Analytical methods for simple models, numerical techniques for complex scenarios.
  \end{itemize}
\end{frame}




 
\end{document}

