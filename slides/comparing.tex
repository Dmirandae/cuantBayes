
\documentclass{beamer}

\title{Hierarchical Tests, hLrT, and Bayes Factors}
\author{DRME}
\date{2024 01 20}

\begin{document}

\begin{frame}
  \titlepage
\end{frame}

\begin{frame}{Hierarchical Tests}
  \begin{itemize}
    \item Hierarchical tests involve a sequence of tests organized in a hierarchical structure.
    \item They are designed to address complex hypotheses by breaking them down into simpler, nested components.
  \end{itemize}
\end{frame}

\begin{frame}{Example: Hierarchical Tests}
  \begin{itemize}
    \item Consider testing the overall effectiveness of a treatment.
    \item Hierarchical tests can break this down into sub-tests, such as testing for treatment effect in different subgroups.
  \end{itemize}
\end{frame}

\begin{frame}{Hierarchical Likelihood Ratio Tests (hLrT)}
  \begin{itemize}
    \item hLrT is a specific type of hierarchical test that uses likelihood ratio tests.
    \item It provides a systematic way to test nested hypotheses in a hierarchical fashion.
  \end{itemize}
\end{frame}

\begin{frame}{hLrT Procedure}
  \begin{enumerate}
    \item Start with the full model representing the most complex hypothesis.
    \item Test nested hypotheses by comparing the likelihood of the nested model to the likelihood of the full model.
    \item Continue testing nested models until the simplest hypothesis is reached.
  \end{enumerate}
\end{frame}

\begin{frame}{Example: hLrT}
  \begin{itemize}
    \item Apply hLrT to assess the significance of various components in a regression model.
    \item Stepwise test nested hypotheses related to individual predictors or groups of predictors.
  \end{itemize}
\end{frame}

\begin{frame}{Bayes Factors}
  \begin{itemize}
    \item Bayes Factors provide a measure of the evidence for one hypothesis over another.
    \item They involve comparing the likelihood of the data under different hypotheses, incorporating prior beliefs.
  \end{itemize}
\end{frame}

\begin{frame}{Interpretation of Bayes Factors}
  \begin{itemize}
    \item A Bayes Factor greater than 1 favors one hypothesis over another.
    \item A Bayes Factor less than 1 favors the other hypothesis.
  \end{itemize}
\end{frame}

\begin{frame}{Example: Bayes Factors}
  \begin{itemize}
    \item Apply Bayes Factors to compare two competing models for explaining a phenomenon.
    \item Evaluate the evidence in favor of a simpler model over a more complex one.
  \end{itemize}
\end{frame}


\end{document}
