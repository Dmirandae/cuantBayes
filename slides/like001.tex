

\begin{frame}{Probability Distribution vs. Likelihood Function}
  \begin{itemize}
    \item Key distinction: Treatment of parameters and data.
    \item In a probability distribution:
      \begin{itemize}
        \item Parameter is fixed.
        \item Data are random variables.
      \end{itemize}
    \item In a likelihood function:
      \begin{itemize}
        \item Data are fixed.
        \item Parameters are variable.
      \end{itemize}
  \end{itemize}
\end{frame}

\begin{frame}{Examining the Relationship Through Plots}
  \begin{itemize}
    \item Probability density function with a fixed $\theta$ and varying $y$:
      \begin{itemize}
        \item Area under the curve equals 1 (Figure 4.2.1A).
      \end{itemize}
    \item Probability density function with fixed $y$ and varying $\theta$:
      \begin{itemize}
        \item Likelihood profile obtained (Figure 4.2.1B).
        \item Area under the curve does not equal 1.
      \end{itemize}
  \end{itemize}
\end{frame}

\begin{frame}{Relationship for Discrete Data}
  \begin{itemize}
    \item Probability mass function $y|\theta$ with fixed $\theta$ and varying $y$:
      \begin{itemize}
        \item Sum of probabilities equals 1 (Figure 4.2.1C).
      \end{itemize}
    \item Probability mass function with fixed $y$ and varying $\theta$:
      \begin{itemize}
        \item Likelihood profile generated (Figure 4.2.1D).
        \item Sum of probabilities does not equal 1.
      \end{itemize}
  \end{itemize}
\end{frame}

\begin{frame}{Units of the Likelihood Profile}
  \begin{itemize}
    \item Units on the y-axis are arbitrary.
    \item Can be scaled to any quantity.
    \item Likelihood profile often scaled so that the peak equals 1.
    \item Scaling achieved by dividing all likelihoods by the maximum likelihood.
    \item Does not alter the relationship between likelihood and probability.
  \end{itemize}
\end{frame}

\begin{frame}{Illustration Through Figure 4.2.1}
  \begin{itemize}
    \item Figure 4.2.1 visually illustrates the subtle distinction.
    \item Parameter $\theta$ is not fixed in the likelihood framework.
    \item Likelihood functions do not define the probability or probability density of $\theta$.
  \end{itemize}
\end{frame}

\begin{frame}{Notation Considerations}
  \begin{itemize}
    \item Some authors use $L(\theta; y)$ instead of $L(\theta|y)$.
    \item Emphasizes that likelihood functions focus on the dependence of parameters on fixed data.
  \end{itemize}
\end{frame}

\begin{frame}{Remarks}
  \begin{itemize}
    \item Probability distributions and likelihood functions highlighted through plots.
    \item Probability and likelihood coincide {\bf{ONLY}} when parameters are treated as fixed.
  \end{itemize}
\end{frame}


